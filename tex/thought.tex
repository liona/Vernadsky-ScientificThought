\documentclass[twocolumn]{book}

\usepackage{etex} % Load this first.
\usepackage{url}
\usepackage{graphicx}
\usepackage[T2A]{fontenc}
\usepackage[utf8]{luainputenc}
\usepackage{csquotes}
\usepackage[russian,french,german,english]{babel}
\usepackage{bigfoot} % install {ncctools} and {bigfoot}

% Font configuration:
%\usepackage[sc]{mathpazo}
%\linespread{1.05}  % line spacing for Palatino.
%\usepackage[scaled=0.95]{helvet}
%\usepackage{courier}

\usepackage[backend=biber,sortlocale=ru,bibencoding=utf8,citestyle=authortitle-trans]{biblatex}
%\usepackage{multicol} % We use up all of TeX's counters with this package.

\usepackage{tocloft}


%\usepackage[pdftex,colorlinks=true]{hyperref} % Load last.
% Add option 'draft' when debugging hyper-links accross page breaks:
\usepackage[colorlinks=true,pdfencoding=auto]{hyperref}


% Load bibliography databases:
\addbibresource{thought.bib}



% Typeset titles of works:
\newcommand{\rtitle}[1]{\emph{#1}}

% Typeset (e.g.) Latin phrases:
\newcommand{\lphr}[1]{\textsf{\emph{#1}}}

% Typeset chapters with by-line-type subtitles:
\newcommand{\ChapterByLine}[2]{\chapter[#1]{#1\\[1ex]\sc\large #2}}

% Typeset chapters:
\newcommand{\Chapter}[1]{%
	\refstepcounter{chapter}%
	\chapter*{Chapter \thechapter%
	  \begin{center}%
	    \rule{0.5\linewidth}{0.1mm}\\
	    \it\large #1\\[-1ex]%
	    \rule{0.25\linewidth}{0.1mm}
	  \end{center}%
	}%
	\chaptermark{}%
	\addcontentsline{toc}{chapter}{Chapter \thechapter}%
	\cftchapterprecistoc{#1}%
}


%%
% Footnotes:
\DeclareNewFootnote{default} % Author's footnotes.
\DeclareNewFootnote{Ed}[fnsymbol] % The editor's footnotes.
\DeclareNewFootnote{Transl}[Roman] % Our (translator's) footnotes.
\MakePerPage{footnoteTransl}
\newcommand{\footnoteRus}[1]{%
	\footnoteTransl{%
	  \begin{otherlanguage}{russian}%
	    #1%
	  \end{otherlanguage}%
	}}

% Use when running pdflatex to generate the font cache, instead of the above
% (when avoiding running out ouf TeX counters by disabling <bigfoot>):
%\newcommand{\footnoteEd}[1]{\footnote{#1}}
%\newcommand{\footnoteTransl}[1]{\footnote{#1}}
%\newcommand{\footnotemarkTransl}{\footnotemark}
%\newcommand{\footnotetextTransl}{\footnotetext}


\title{Scientific Thought as a Planetary Phenomenon}
\author{Vladimir Ivanovich Vernadsky \and In translation by the LaRouche Movement.}


\begin{document}

\selectlanguage{english}

\frontmatter

\begin{titlepage}

\begin{center}
\Huge\textbf{Scientific Thought as a Planetary Phenomenon}


\bigskip
\LARGE{V. I. Vernadsky}

\Large{1936--1938}


\bigskip
\Large{Translated from Russion by Pavel M. Penev for the LaRouche movement.}
\foreignlanguage{russian}{\nocite{vernadsky1938thought-lib.ru}}

October 20, 2010--present

\vfill

\copyright\ 2012 Pavel M. Penev

\end{center}

\end{titlepage}

\tableofcontents
\ChapterByLine{Remarks to the Electronic Edition~\dots}{
V. I. Vernadsky Electronic Archive\\
\url{http://vernadsky.lib.ru}
}

The present electronic edition of V. I. Vendasky's book \rtitle{Scientific
Thought as a Planetary Phenomenon\footnoteRus{Научная мысль как планетное
явление}} was being prepared according to the
edition\fullcite{vernadsky1938thought-yanshin91} at the end of 1999.

The first four chapters were prepared by April, 2000, and added to the Maxim
Moshkov library (\url{http://lib.ru/FILOSOF/WERNADSKIJ/}).  These first
chapters were carefully proofread and, I hope, contain very few printing
errors.

The fifth and sixth chapters were proofread (also quite carefully, though not
as well as the first four) by the end of November, 2000.  They were published
on the server of the Electronic Archive (\url{http://vernadsky.lib.ru}), but
were not sent to the Moshkov library in the hope that the remaining four
chapters would be prepared sufficiently quickly.

Unfortunately, because of insufficient time, the work on the remaining chapters
kept dragging on and on, to the point that I decided to use the electronic
version of these chapters, which was prepared by the Russian Foundation for
Fundamental Research\footnoteRus{Росийским Фондом фундаментальных исследований}
from the edition \fullcite{vernadsky1938thought-feniks97}.

However, comparing these two editions, it seemed to me, that the earlier one,
from 1991, was much closer to the original text of V. I. Vernadsky.  The 1997
edition is filled with slight editorial corrections, which, though nowhere (it
seems) distort Vernadsky's meaning, nevertheless, quite strongly change his
manner of exrpession, and that in such a way that at these places the mind is
often just tripped up, and it is at once apparent that Vladimir Ivanovich could
not have written in that manner.  It is, therefore, necessary to streighten out
chapters 7--10 according to the 1991 edition with time.  It is also necessary
to proofread all chapters once again, and correct any remaining errors.

I include the introductions of the editors of both editions at the begining of
this book, which tell about the history of the writing of Vladimir Ivanovich's
book, as well as about the history of its hard and quite controversial
publication.

For commercial use of the electronic edition of \rtitle{Scientific Thought as a
Planetary Phenomenon}, or (which would be just terrific ;-) for aid with its
proofreading, contact me at the address indicated on the
\url{http://vernadsky.lib.ru} server.

*Note:* The electronic edition is being prepared in the LaTeX format; it is
necessary to update that version, and not the derived HTML version in the Maxim
Moshkov library when proofreading.

\begin{flushright}
Sergey Mingaleev\footnoteRus{Сергей Мингалеев}\\
October 16, 2001
\end{flushright}

\ChapterByLine{Preface and remarks by A.~L.\ Yanshin~\dots}{%
A.~L.\ Yanshin\\
{\rm\normalsize Chairman of the Committe of the Academy of Sciences of the USSR
for the Exploitation of the Scientific Heritage of Academician V. I.
Vernadsky}\footnotemarkTransl
\\[1ex]
F.~T.\ Yanshina\\
{\rm\normalsize Director-founder of the museum home of Academician V.~I.\ 
Vernadsky}}%
\footnotetextTransl{\foreignlanguage{russian}{Комиссия по разработке научного
наследия академика В.~И.\ Вернадского}}%

The electronic version of the preface and the remarks was prepared from the
edition in the book \fullcite{vernadsky1938thought-yanshin91}.

\section*{Preface}

The name of Vladimir Ivanovich Vernadsky has become widely known in our
country.  There is nobody with even the slightest degree of education, who
hasn't read, if not Vernadsky's works, then, at least, numerous newspaper and
magazine articles about him and his work.

There is a Vernadsky Avenue in Moscow.  One of the largest institutes at the
Academy of Sciences of the USSR, the Institute of Geochemistry and Analytical
Chemistry,\footnoteRus{Институт геохимии и аналитической химии} bears his name.
There is a Committee for the Exploitation of the Scientific Heritage of
Academic V.~I.\ Vernadsky, which publishes its own circular, at the Presidium
of the Academy of Sciences of the USSR.  Branches of that Committee work in
Leningrad and in Kiev.  There have been grants under Vernadsky's name
established at Moscow, Leningrad, Kiev, and Simferopol University.  National
scientific centers for the study of the work of this prominent thinker and for
its application to the solution of contemporary problems exist in Odessa,
Rostov-na-Don, Erevan, Simferopol, Ivanov, and in other cities in the USSR, and
abroad---in Prague, Oldenburg and Berlin.\footnote{
	Also named after V. I.  Vernadsky are: the National Geological
	Museum,\footnoteRus{Государственный геологический музей} the National
	Public University of Biospheric Studies,\footnoteRus{Всесоюзный
	народный университет биосферных знаний} the Central Scientific Library
	of the AS UkrSSR,\footnoteRus{Центральная научная библиотека АН УССР}
	the Student Sociological Center ``Noosphere'',\footnoteRus{Студенческий
	социологический центр ``Ноосфера''} the peak in the basin of
	Podkamennaya Tunguska River, the crater on the dark side of the Moon,
	the peninsula in Eastern Antarctica near the Sea of
	Astronauts,\footnoteRus{Море Космонавтов} the forest on the island of
	Paramushir (Kuril Island), the subglacial forests in Eastern
	Antarctica, the underwater volcano in the Atlantic Ocean, the mine in
	the region of Lake Baikal, the mineral Vernadit,\footnoteRus{вернадит,
	$Mn^{4+}, Fe^{3+}, Ca, NaS(O,OH)_{2n}\cdot H_2O$} the diatomaceous
	algae, research vessel ``Academician Vernadsky'' of AS UkrSSR, the
	steamboat ``Geologist Vernadsy''\footnoteRus{Геолог Вернадский} of the
	Kama River Shipping company,\footnoteRus{Камское речное пароходство}
	the Vernadsky village near Simferopol, the Vernadsky railway station on
	the Kazan line, the subway stop ``Vernadsky Avenue'' in Moscow, the
	Biosphere Museum at the Leningrad branch of the Institute of the
	History of Natural Science and Technology of the AS USSR.  A V.~I.\ 
	Vernadsky monument has been erected in Kiev, a memorial plate is in
	place on the old building of Moscow State University M.~V.\ 
	Lomonosov,\footnoteRus{МГУ им.\ М.~В.\ Ломоносова} on Vernadsky Avenue
	in Moscow, on the building of Leningrad State
	University,\footnoteRus{Ленинградского государственного университета}
	as well as on the building of the Kiev State University T. G.
	Shevchenko.\footnoteRus{Киевского государственного университета им. Т.
	Г.  Шевченко}  Bonuses V.~I.\ Vernadsky are awarded for exceptional
	scientific work in the areas of mineralogy, geochemistry and
	astrochemistry by the Academy of Sciences of the USSR and by the
	Academy of Sciences of the UkrSSR.  A golden medal named after him has
	been established by the Academy of Sciences of the USSR.}

V.~I.\ Vernadsky's 125th birthday was celebrated in March 1988 in our country,
as well as abroad (in Prague and in Berlin).

The celebrations srpead very widely.  An exhibition dedicated to his work was
opened on January 15, 1988 at the Exhibition of the Achievements of the
National Economy.\footnoteRus{ВДНХ, from выставка достижений народного
хозяйства}  Scientific symposia on different directions of V.~I.\ Vernadsky's
research took place successively in Leningrad, Kiev, and Moscow with the
participation of foreign scientists from March 3 to 11.  A commemorative
conference took place in Balshoy Theatre\footnoteRus{Большой театр} in Moscow
on his birthaday, March 12, with the participtation of public organizations.
Separate conferences and scientific sessions took place during the same days in
Ivanov, Odessa, Simferopol, Rostov-na-Don, Yerevan, Baku,
Almaty,\footnoteRus{Алма-Ате} Novosibirsk, Irkutsk, and in many other
scientific centers of the nation.  The proposal to create an International Fund
V.~I.\ Vernadsky\footnoteRus{Международного фонда В.~И.\ Вернадского} for
subsidizing the translation of his works in other languages, finding materials
about him in foreign archives, and the invitation of scientists from foreign
nations to the USSR for reports and lectures on the contemporary development of
scientific problems noted by V.~I.\ Vernadsky was accepted.\footnote{See the
information at the end of the book.}  Articles about him, and his multifaceted
scientific work have appeared in almost all Soviet and international newspapers
and magazines.

. . .



\mainmatter

%\Chapter{}
\refstepcounter{chapter}

\Chapter{%
Manifestation of the historical moment mankind is currently living through as a
geological process.  Evolution of the species of living matter and evolution of
the biosphere into the noosphere.  This evolution cannot be stopped by the
course of global human history.  Scientific thought and mankind's daily lives
as expressions of it.}

14. We are not yet conscious of, we are not yet living the realization of the
full consequences of the astonishing, unprecedented times that mankind has
entered during the 20th century.

We are living at the threshold of an extremely important, fundamentally new
epoch in the existence of mankind, in mankind's history on our planet.

Mankind has, for the first time, encompassed the whole surface envelope of the
planet---the whole biosphere, all parts of the planet connected to life---with
human life, with human culture.

We are present at, and are actively participating in the creation of a new
\emph{geological factor} in the biosphere, unprecedented in its power and in
its unity.

It has been scientifically established for the last 20--30 thousand years, but
has been clearly manifested at an ever increasing rate only during the last
millenium.

The envelopment of the whole surface of the biosphere by a unified social
species of the animal kingdom---by \emph{mankind}---has been completed after
many hundreds of thousands of years of unstoppable, tempestuous striving for
it.  There is no corner on Earth inaccessible to mankind.  There is no limit to
our possible population growth.  Man, through scientific thought and through
his life, socially organized into states, and guided by technology, is creating
a new \emph{biogenic force} in the biosphere, which is guiding his population
growth and creating favorable conditions for his population in parts of the
biosphere, earlier impenetrable to human life, and even in places where there
was no life before.

Theoretically, we cannot foresee a limit to mankind's potential, if we only
take into account the effect of generations; every geological factor is fully
manifested in the biosphere only in the effect of generations of living beings,
only in geological time.  With the rapidly increasing precision of scientific
work---in this case, of the methodology of scientific observation,---we can now
clearly establish, and study the increase of this new, principally currently
emerging, geological force in historical time.

Mankind is a unified whole, and even if that is recognized by the vast
majority, this unity manifests itself in forms of human life, which actually
deepen and strengthen it without being noticed by man, impetuously, [as a
result of] an unconscious striving for it.  Human life, with all of its
variety, has become indivisible, unified.  An event, ocurring in a forsaken
corner on land or in the ocean, is reflected, and has consequences, major or
minor, in a multitude of other places, all over the Earth.  The telegraph,
telephone, radio, airplanes, aerostats\footnoteTransl{An aerostat is an object
that can stay stationary in air, bacause it is lighter than it, such as a
baloon or a dirigible.} encompass the globe.  Communication is ever easier and
faster.  Its organization increases, turbulently grows, every year.

We can clearly see that this is the beginning of a tempestuous movement, of a
natural phenomenon, which cannot be stopped by the accidents of human history.
Here the relation between historical processes and the paleontological history
of the manifestation of Homo sapiens is expressed, maybe for the first time.
That process---\emph{the complete colonization of the biosphere} by
mankind---arises from the course of the history of scientific thought, which is
inseparably connected with the speed of communication, with the achievements of
transportation technology, with the ability of thoughts to be communicated
\emph{instantaneously}, and to be discussed everywhere on the planet
simultaneously.

The fight, which is being carried out against this main historical current, is
forcing even its ideological opponents to obey it.  Government formations,
ideologically rejecting the equality and unity of all people, are attempting,
lacking no resources, to halt its impetuous manifestation; but it can hardly be
doubted that these utopian dreams would fail to last.  This transformation will
inevitably come to pass in the course of time, sooner or later, since the
creation of the noosphere out of the biosphere is a natural phenomenon,
fundamentally deeper and more powerful than human history.  It necessitates the
manifestation of mankind as a unified whole.  This is its inevitable
requirement.

Ours is a new stage in the history of the planet, which does not allow
comparison with past history without corrections.  It is so, because this stage
is creating fundamental \emph{novelty} in the history of the whole Earth, and
not just in the history of mankind.

Man has actually recognized for the first time that he is a citizen of the
\emph{planet} and that he can---must---think and act in a new aspect, not only
in the aspect of individual personalities, nuclear or extended families,
nations or their unions, but also in a \emph{planetary aspect}.  He, like
everything living, can think and act in a planetary aspect only in the region
of life---in \emph{the biosphere}, in a certain earth envelope, with which he
is inseparably and lawfully connected, and outside of which he cannot go.  His
existence is a function of it.  He carries it everywhere with himself.  And he
inevitably changes it lawfully and unceasingly.


15. Simultaneously with mankind's complete envelopment of the surface of the
biosphere---with its complete colonization,---which is closely connected
with the achievements of scientfic thought, i.e. with the course of scientific
thought in time, a scientific generalization, which scientifically reveals the
character of the historical moment mankind is currently living through in a new
way, has been formed in \emph{geology}.

Mankind's geological role has been cast anew in the understanding of
geologists.  True, the recognition of the geological significance of our social
life has been expressed in a less clear form long ago, much earlier in the
history of scientific thought.  However, at the beginning of our century C.
Schuchert [1858--1942] in New
Haven,\footnote{\fullcite{schuchert1933geology-p80}} and A.~P.\ Pavlov
(1854--1929) in
Moscow\footnote{\foreignlanguage{russian}{\fullcite{pavlov1936geologicheskaya-p105}}}
independently accounted, geologically anew, for the long-known change which the
emergence of human civilization introduces into the environment, onto the face
of the Earth.  They considered it possible to take this manifestation of Homo
sapiens as the basis for distinguishing \emph{a new geological epoch,} along
with the tectonic and orogenic data which usually determine such divisions.

They correctly tried to split the Pleistocene Epoch, defining its end by the
beginning of the manifestation of mankind (during the recent hundred-somethng
thousand years---say a few decamyriads ago), and separating the latter in its
own geological epoch: \emph{psychozoic,} according to Schuchert;
\emph{anthropogenic,} according to A.~P.\ Pavlov.

Actually Ch.\ Schuchert and A.~P.\ Pavlov deepened and made more precise,
brought into the established in modern geology divisions of the history of the
Earth, a conculsion, which was made much before them, and which did not
contradict the empirical scientific work.  This conclusion was clearly
recognized by one of the creators of contemporary geology, L. Agassiz
(1807--1873), based on the paleontological history of \emph{life}.  He
established the special geological \emph{epoch of mankind} already in 1851.

However, Agassiz relied not on geological facts, but rather, to a great extent,
on the common religious conviction so strong during the age of natural science
before Darwin; he started from the special position of man in the
universe.\footnote{Agassiz expressed that idea in a polemical work directed
against Darwinism (\fullcite{agassiz1859essay}).  It is possible that this is
related to why the work did not reach, [despite] the many important reflections
in it, the influence it could have had.}

The geology in the middle of the 19th century, and the geology at the beginning
of the 20th century are incomparable in their power and scientific
justification, and the epoch of mankind of Agassiz is not scientifically
comparable with the epoch of Schuchert-Pavlov.

Already earlier, when geology was just being created and its basic concepts did
not yet exist, G.\ Buffon (1707--1788) notably expressed that same geological
epoch of mankind at the end of the 18th century.  He proceeded from the ideas
of the philosophy of the Enlightenment, advancing the significance of reason in
the conception of the universe.

The definite difference between these homonymous concepts is clear from the
fact that Agassiz assumed the geological age of the World to be the biblical
duration of the existence of the Earth---six--seven thousand years,---Buffon
thought about an age of more that 127 thousand years, Schuchert and Pavlov---of
more than a billion years.


16. We have already met with similar conceptions in philosophy long ago.
Conceptions, which have been reached in another way---not by way of precise
scientific observation and experimentation, like that of C. Schuchert, A. P.
Pavlov, L. Agassiz (and J. Dana, who knew about the generalizations of
Agassiz), but by way of philosophical searches and intuition.

The philosophical worldview creates, in general, as well as in particular, that
environment, in which scientific thought takes place and develops.  To a
significant extent, it determines and gives rise to scientific thought, itself
being changed by its achievements.

The philosophers relied on free, it seemed to them, in their expression ideas,
on the searches of confused human thought, of human consciousness, which
wouldn't reconcile with reality.  However, man unavoidably built his ideal
world in the brutal framework of surrounding nature, the environment of his
life, the biosphere, with which he has a deep connection, independent of his
will, which he did not, and still does not, understand.

We find, in the history of philosophy, already many centuries before our age,
intuitions and constructs, which could be connected to scientific empirical
conclusions, if we translate the thoughts---intuitions---that have reached us
into the realm of real scientific facts of our time.  We lose their roots in
the past.  A few of the philosophical searches in India, many centuries
ago,---the philosophy of the Upanishads---can be interpreted in such a way, if
we translate them into the realm of 20th century science.\footnote{The
philosophy of The East, mainly of India, in connection with the new creative
work there, taking place under the influence of the introduction of Western
science in Indian culture, is of much greater interest for life sciences than
Western philosophy, which is deeply permeated---even in its materialistic
parts---by deep echoes of Judeo-Christian religious searches.}

Analogous conceptions existed in another, smaller, cultural area, partly
overlapping, but later, which was isolated from the Indian one for a
significant part of the time: in the circle of the Helenic Mediterranean
civilization.  We can trace the germs of these conceptions going back almost
two and a half thousand years ago.  The significance of science and scientists
for the government of the polis in political and social thought is clearly
manifested in Helenic thought, and is notably expressed in the concept of the
sate, [given by] Plato [427--347].

It cannot, it seems, be denied, but the condition of the sources, reaching us
in fragments, also does not allow us to confirm precisely, that after Aristotle
[384--322] these ideas were still alive during the Helenic age of Alexander the
Great [356--323], when, a few centuries after the destruction of the Persian
kingdom, a close exchange of ideas and knowledge between Helenic and Indian
civilization was established.  A connection between them and Chaldean
scientific thought, which went back a few millenia before Helenic and Indian
thought, was established at the same time.  The history of scientific work and
thought during this remarkable age is just beginning to come to light.

Better known is the influence of the Helenic political and social ideas.  We
can trace their historical influence exactly in the historical process of
modern science and of the civilization of the European West, which replaced the
theocratic ideological structure of the Middle Ages.  We can see their growth
in pactice, and with clarity only during the 16th--17th centuries, in the
conceptions and constructs of F. Bacon (1561--1626), who prominently advanced
the idea of the power of man over nature as the aim of modern science.

In the 18th century, in 1780, G.\ Buffon posed the manifestation of man's
control of nature \emph{as part of the history of the planet} not as an idea,
but as an observable natural phenomenon.  He relied on the hypothetical
reconstruction of the planet's past, connected with philosophical intuition and
theory, rather than on precisely observed facts---but he was looking for them.
His ideas were adopted by philosophical and political thought, and,
undoubtedly, exerted their influence on the course of scientific thought.
Geologists from the end of the 18th--beginning of the 19th century often relied
on them in their current scientific work.


17. The scientific constructs of Schuchert and Pavlov and all the scientific
work which---to a significant degree unconsciously---preceded them are
essentially distinct from these philosophical constructs, which, however (this
can be established historically), undoubtedly influence the course of
geological thought, though unable to give it a firm basis.

It is clear from the generalizations of Schuchert and Pavlov that the main
influence of human thought as a geological factor is expressed in its
scientific manifestation: it mainly builds and guides the technical work of
mankind, which is transforming the biosphere.

Both of the indicated geologists were able to make their generalizations,
above all, because mankind was able to colonize the whole planet in their
time.  No organism except him, save for microscopic species and, possibly, a
few graminoids, has encompassed such an area in populating the planet.
However, mankind has accomplished this in a different way.  He thought
scientifically and transformed the biosphere through labor, adapted it to
himself and himself created the conditions for the manifestation of his
characteristic biogeochemical energy of reproduction.  Such population of the
whole planet became clear at the beginning of the 20th century, and it could
be considered a fact since about the first quarter of that century, which is
being confirmed every year in front of our eyes.  It became possible only
thanks to the drastic change of the conditions of life connected with the
emergence of a new ideology, with the drastic change in the tasks of
government life, with the scientific growth of technology, which were being
carried out at the very same time.

As J. Ortega y Gasset\footnote{\fullcite{ortegaygasset1932revolt-p19}}
correctly remarked, the 19th century in Europe, and over the whole world since
its second half, was a historical period when the significance of the vital
interests of the masses of population occupied first place in practice and
ideology in their consciousness and in the consciousness of government people
for the first time in wold history.  It was dramatically manifested in everyday
life for the first time.  A new ideology was based on the consciousness of the
population masses stepping onto the historical stage as a social force for the
first time.  It is beginning to encompass all mankind---every language without
exception---at a rapidly increasing rate.

It will show in its real significance only in the course of time.

The social-political ideological shift was dramatically manifested in the 20th
century mainly thanks to scientific work, thanks to the scientific
determination and clarification of the social tasks of mankind, and of the
form of his organization.


18. The question of the better organization of life and of the means by which
it could be accomplished has been raised numerous times during the
multi-thousand-year historical tragedy full of blood, suffering, crime,
destitution, hardship, which we call world history.  Man did not accept the
conditions of his life.

The exit from these searches has been resolved differently, and we can see
numerous (and how many have disappeared without trace!)
searches---philosophical, religious, artistic and scientific.  For millenia
they have been, and are being created in every corner where human society has
existed.

The world history of mankind has been lived and recreated for a significant
part of the human population, and the places and times full of suffering,
evil, slaughter, hunger, and destitution for the majority have been an
unsolvable mystery from a \emph{human} point of view of sensibility and
goodness.  In general, innumerable philosophical and religious attempts during
the course of millenia have not reached a unified explanation.

All solutions reached in such a way transfer and have transferred the question
in a different plane---from the domain of heard reality into the domain of
ideal constructs.

. . .

\Chapter{%
The movement of scientific thought in the 20th century, and its significance in
the geological history of the biosphere.  Its main characteristics: explosion
of scientific work, change in the understanding of the fundamentals of reality,
ecumenicism, and efficient, social manifestation of science.}

47. What is presently occurring in the scientific movement can only be compared
with that scientific movement from the past of science, which was connected
with the birth of Greek philosophy and science in the 6th--7th c.~BC.

Unfortunately, so far we cannot clearly imagine that accumulation of scientific
knowledge which the ancient Greeks had amassed at the time when scientific
thought manifested itself in their environment, and when it, for the first
time, acquired a scientific-philosophical structure, outside of religious,
cosmogonic and poetical constructs---when the scientific method was created for
the first time in the Hellenic city civilization of the polis---logic and
theoretical mathematics applied to life, when the search for scientific truth
became a reality, as a goal for itself in the life of the individual in a
social environment.

The circumstances of this, as history has shown, momentous event in mankind's
life, and in the evolution of the biosphere are, to a large extent, mysterious
and the history of scientific knowledge is being clarified slowly, but
nevertheless ever deeper.  Clear is only a general sketch of the accumulation
of scientific knowledge of the Hellenic environment at that time, the
achievements of the thinkers of Hellenic science, who lived at the time, and
what they received from the previous generations of Hellenic civilization.  We
are slowly beginning to understand this.  This is on the one hand.

And on the other hand, the conceptions about what the Greeks received from
great civilizations preceding them---Asia Minor, Cretan, Chaldean
(Messopotamian), Ancient Egypt, India---are now starting to drastically
change.

Unfortunately, only a \emph{miniscule part} of Hellenic scientific literature
has reached us.  The major researchers have left no trace in the literature
accessible to us, or only fragmentary indications of their scientific work has
reached us.

True, a large part of the complete works of Plato has reached us, as well as a
significant part of Aristotle's scientific works, however, many of the latter's
works, fundamental from the standpoint of the scientific search, have been
lost.  Especially unfortunate, from this standpoint, is the loss of the works
of major scientists, in whose output scientific thought and the scientific
method entered the age of flourishing and synthesis of Hellenic
science---Alcmaeon (500~BC), Leucippus (430~BC), Democritus (420--370~BC),
Hippocrates of Chios (450--430~BC), Philolaus (5th century BC) and many others,
from whom only miniscule fragments, or nothing but names have remained.

The loss of the first attempts at histories of scientific work and thought,
which were written closest to the centuries of its manifestation, may be even
more unfortunate.  Partly distorted, and in an incomplete form, this work has
reached us in the form of nameless essentials, sometimes adapted and skewed in
the course of the many centuries after their publication.  But the originals of
Xenocrates' (397--314) history of Geometry, Eudemus of Rhodes' (circa~320)
history of science, Theophrastus' (372--288) historical books, and others have
been lost in the historical course of Greko-Roman civilization by the time of
our age---during the centuries closest to it, almost a thousand years ago.

In essence, the basic fund of Helenic science---what I call a \emph{scientific
apparatus}\footnote{
	\foreignlanguage{russian}{\fullcite{vernadsky1939problemy2-p9-10}
	(Problems of Biogeochemistry II)}
}---has reached us in miniscule fragments, passing, on top of it, through many
centuries, in the remains of Aristotle's and Theophrastus's works on the
history of natural sciences, as well as in the works of Greek mathematicians.
Nevertheless, it exerted tremendous influence on the Renaissance and on the
creation of Western European science in the 15th--17th centuries.  Our modern
science has been created, to a significant extent, relying on and starting from
this fund's achievements, developing the ideas and knowledge laid out in it.
Broken for centuries, that already during the time of the Roman Empire, the
threads were restored in the 17th century.


48. The recent course of the history of science requires us to change
our conceptions of that pre-Hellenic heritage, from which Hellenic science
sprouted, as I already indicated (§42).

The Greeks have everywhere pointed to the great knowledge, which they had
received from Egypt, Chaldea, the East.  We must now admit that they were
correct.  Science had already existed before them---the science of the
``Chaldeans'', reaching back beyond millenia BC, is only now being uncovered
before us---in fragments, proving beyond any doubt its long unsuspected, until
our time, force (§42).

It is now becoming clear that we must attribute a much more real significance,
than has been recently done, to the numerous indications by ancient scientists
and writers of the fact that the creators of Hellenic science and philosophy
took into consideration, proceeded in their creative work from the achievements
of scientists and thinkers from Egypt, Chaldea, Arian and non-Arian
civilizations of the East.

Babylonian scientists worked together with Greek ones in the course of several
centuries.  At the same time, the new flourishing of Babylonian astronomy
occurred in the centuries closest to our age.  Gradually, in the course of
several generations, they merged into the Hellenic cultural environment and
equally suffered the unfavorable for science circumstances of that time (§40).
Undoubtedly, the knowledge received from the scientists of that time was used
by the Greeks during the period of this dialogue.

Undoubtedly, what was harnessed and used by them was very significant by that
time---especially if we consider the multimillenial experience and the
multimillenial tradition of seafaring, engineering, agriculture, irrigation
works, military art, government organization and everyday life.

For centuries Greek science worked in direct contact with Chaldean and Egyptian
science, was merging with them.  Though it is possible that creative thought in
Egyptian science died out during that time---this didn't happen with Chaldean
science (§42).

Hellenic science, in the age of its birth, is a direct continuation of the
intense creative thought of pre-Hellenic science.  This fact is acknowledged,
but still not assimilated, in the history of science.

The ``miracle'' of Hellenic civilization---a historical process, whose results
are clear, but whose course cannot be precisely traced---was a historical
process like others.  It had a solid basis in the past.  Only its result in its
achievement---the rate at which it was achieved---turned out to be singular in
time, and exceptional in its consequences in the noosphere.


49. 

%\Chapter{}
\refstepcounter{chapter}

%\Chapter{}
\refstepcounter{chapter}

%\Chapter{}
\refstepcounter{chapter}

\Chapter{%
The structure of scientific knowledge as a manifestation of the noosphere, the
geologically new state of the biosphere resulting from this knowledge.  The
historical course of the planetary manifestation of Homo sapiens by means of
its creation of a new form of cultural biogeochemical energy, and the noosphere
associated with it.}

100. The sciences of the biosphere and its objects, i.e. all humanities
without exception, natural sciences, in the term's own meaning, (botany,
zoology, geology, mineralogy, etc.), all engineering sciences---applied
sciences in the general meaning of the term---are areas of knowledge, which are
maximally accessible to mankind's scientific thought.  Here millions of
millions of incessantly scientifically established and systematized facts,
which are the results of organized scientific work, are concentrated, and are
unstoppably increasing, quickly and consciously, with every generation,
beginning with the 15th--17th centuries.

In particular, the scientific disciplines of the constitution of means of
scientific knowledge, inseparable from the biosphere, can be viewed
scientifically as a geological factor, as a manifestation of the biosphere's
organization.  These are sciences ``of the spiritual'' work of the human
individual in one's social environment, sciences of the brain and organs of
sense, the problems of psychology and logic.  They give rise to the search for
the fundamental laws of human scientific knowledge, that power which has, in
our geological age, transformed the biosphere encompassed by mankind into a
natural body, new in its geological and biological processes---into a new
state, into the noosphere,\footnote{
	\foreignlanguage{french}{\fullcite{leroy1928origines-p37-57}}
} to whose consideration I shall return below.\footnoteEd{
	See \foreignlanguage{russian}{\fullcite{vernadsky1987himicheskoe-ch21}}}

Its emergence in the history of the planet, beginning intensively (on the
scale of historical time) a few tens of thousands of years ago, is an event of
great importance in the history of our planet, connected, in the first place,
with the growth of sciences about the biosphere, and is, obviously, not
accidental.\footnote{
	I will return to this process later.  Here I only note Le Roy's thought
	(1928): \foreignquote{french}{Deux grands faits, devant l'esquels tous
	les autres samblent presque svanouir, dominent dans l'histoire passe de
	la Terre: la vitalisation de la matire, puis l'hominisation de la
	vie.}---Op. cit., p.47.  \enquote{Two major facts, in comparison to
	which all others seem almost unnoticeable, predominate in the history
	of the Earth: the vitalization of matter, and the humanization of life.
	The first one is hypothetical, but the beginning of the second is
	clearly visible.}}

We can say that, in this manner, the biosphere is the main area of scientific
knowledge, even if we are only now beginning to differentiate it scientifically
from our surrounding reality.


101. It is clear from what has been said, that the biosphere corresponds to
that, which in the thought of naturalists and in most of philosophical
thought, in the cases where they were not concerned with the Cosmos as a
whole but remained within the limits of the Earth, corresponds to Nature as
usually understood, the Nature of the naturalist in particular.

However, this nature is not amorphous and shapeless, as it has been considered
for centuries, but has definite, very precisely delineated
structure,\footnote{
	This ``structure'' is very peculiar.  It is not a mechanism or anything
	motionless.  It is dynamic, always variable, moving, changing at every
	moment, and never returning to a previous type of equilibrium.  It is
	closest to a living organism, differing, however, from it in the
	physical-geometrical state of its space.  The space of the biosphere is
	physically-geometrically inhomogeneous.  I think that it is convenient
	to define this structure by means of a special concept of organization.
	See \fullcite{vernadsky1934problemy1, vernadsky1980problemy}.
} which must, as such,
be reflected, and considered in all conclusions and results concerning Nature.

It is especially important in scientific research that this is not forgotten
and that it is taken into account, since unconsciously, opposing the human
individual to Nature, the scientist and thinker gives in to the greatness of
Nature above the human individual.

But life in all of its manifestations, the manifestation of the human
individual included, radically changes the biosphere in such a degree that not
only the agglomeration of indivisible units of life, but, in a few problems,
also the single human individual in the noosphere could not be left without
attention in the biosphere.


102. Living nature\footnoteTransl{
	A literal translation of the Russian expression for the living part of
	nature.
} is a main characteristic of the manifestation of the biosphere, it is the
very distinction of the biosphere from the other earth envelopes.  The
structure of the biosphere is characterized, first of all, and most of all, by
life.

We shall see further on (§135) that between the physical-geometrical
properties of living organisms---they are manifested in the form of their
agglomerations in the biosphere---living matter, and those properties of inert
matter, which constitutes the dominant part of the biosphere by weight and by
number of atoms, there is in several respects an impassible gulf.  Living
matter is a carrier and creator of free energy absent from any other earth
envelope on such a scale.  This free energy---biogeochemical
energy\footnote{
	The concept of biogeochemical energy was introduced by me in 1925 in a
	still-unpublished report to the R.~Rosenthal fund in Paris. (The fund
	does not exist any more.)  This fund gave me the ability to work
	without interruption for two years.  The concept has been presented by
	me in print in numerous articles and books:
	\begin{itemize}
	  \item \foreignlanguage{russian}{\cite{vernadsky1926biosfera-p30-48}};
	  \item \foreignlanguage{french}{\cite{vernadsky1926etudes1,
		  vernadsky1927etudes2}};
	  \item \foreignlanguage{russian}{\cite{vernadsky1926razmnozhenii1},
		  \cite{vernadsky1926razmnozhenii2}};
	  \item \foreignlanguage{french}{\cite{vernadsky1926multiplication1,
		  vernadsky1926multiplication2}};
	  \item \foreignlanguage{russian}{\cite{vernadsky1927bakteriofag}}.
	\end{itemize}
	[Ed.:] For the R.~Rosethal fund's report
	\foreignlanguage{russian}{\rtitle{Живое вещество в биосфере}} see:
	\foreignlanguage{russian}{\cite{vernadsky1994zhivoe-p555-602}}
}---encompasses the whole biosphere and generally determines all of its
history.  It gives rise to and sharply changes the intensity of the migration
of chemical elements constituting the biosphere, and determines their
geological significance.

A new form of this energy, even greater in its intensity and complexity, has
been created and has been quickly increasing in its significance in the domain
of living matter during the last ten thousand years.  This new form of energy,
connected with the activity of human societies, of the genus Homo and others
(Hominidae) close to it, preserves the manifestation of the usual
biogeochemical energy, but at the same time gives rise to a new kind of
migration of chemical elements, leaving, in its variety and power, the usual
biogeochemical energy of living matter on the planet far behind.

This new form of biogeochemical energy, which can be called energy of human
culture, or cultural biogeochemical energy, is the form of biogeochemical
energy, which is presently creating the noosphere.  Later on I shall return to
a more detailed presentation of our knowledge of the noosphere and its
analysis.  But it is now necessary to sketch its manifestation on the planet.

This form of biogeochemical energy is characteristic not only of Homo sapiens,
but also of all other living
organisms.\footnote{
	\foreignlanguage{russian}{\cite{vernadsky1926biosfera-p30-48}}.  See
	\foreignlanguage{russian}{\cite{vernadsky1994zhivoe-p330-341,
	vernadsky1926razmnozhenii1, vernadsky1926razmnozhenii2}}.  Published
	under the title \foreignlanguage{russian}{\rtitle{О размножении
	организмов и его значении в строении биосферы}} in the book
	\foreignlanguage{russian}{\cite{vernadsky1992trudy-p75-101}}.
}  It is, however, negligible in them in comparison to the usual biogeochemical
energy, and has a hardly noticeable effect on the balance of nature, and that
only in geological time.  It is connected to the psychological activity of
organisms, to the development of the brain in the highly developed
manifestations of life, and is expressed in a form resulting in the
transformation of the biosphere into a noosphere only with the emergence of the
human mind.

Its manifestation in mankind's predecessors has been produced, apparently,
over hundreds of millions of years, but it could be expressed in the form of a
geological force only in our time, when Homo sapiens has encompassed with our
life and cultural work the whole biosphere.


103. The biogeochemical energy of living matter is determined, above all, by
the reproduction of organisms, and by their inevitable tendency, determined by
the energetics of the planet, toward a minimum of free energy---it is
determined by the fundamental laws of thermodynamics, corresponding to the
existence and stability of the planet.

It is expressed in the respiration and feeding of organisms---``laws of
nature'', which have not been discovered in their mathematical expression to
this day, but the task of searching for whose expression was clearly laid out
already in 1782 by C. Wolf at the St. Petersburg Academy of
Sciences\footnoteRus{Петербургской Академии наук} at the time.\footnoteEd{
	\cite{vernadsky1954sochineniya-p50}}

Obviously, this biogeochemical energy, in this form, is characteristic of Homo
sapiens, as well.  It is, as with all other living organisms, a species
characteristic,\footnote{
	On the species charactestic see \cite{vernadsky1930considerations}.
} and seems unchangeable to us in the course of historical time.  The other
form, of ``cultural'', biogeochemical energy is also unchanging, or hardly
changing for other organisms.  This other form is expressed in the everyday and
in the technical conditions of organisms' life---in their movement, in their
daily activity and construction of dwellings, in the transportation of their
surrounding matter, etc.  It, as I have already indicated, comprises a
negligible fraction of their biogeochemical energy.

With mankind, this form of biogeochemical energy, associated with the human
mind, grows and increases in the course of time, quickly taking first place.
This growth is possibly related to the growth of the mind itself---apparently,
a very slow process (if it, in fact, occurs at all)---but mainly---with the
increase of the precision and depth of its use, associated with the conscious
change of the social setting, and, particularly, with the growth of scientific
knowledge.

I shall proceed from the fact that the skeletons of Homo sapiens, including
the skull, over a hundred millenia gives us no basis for viewing them as
belonging to another species of man.  This is admissible only under the
condition that the brain of Paleolithic man does not differ in any significant
degree in its structure from the brain of contemporary man.  At the same time,
there is no doubt that the mind of that man from the Paleolithic for this
species of Homo cannot bear comparison to the mind of contemporary man.
Thence it follows that the mind is a complex social structure, built, for the
man of our times, just as for the Paleolithic man, upon the same nervous
substrate, but in a different social setting, which is being composed through
time (space-time, in essence).

Its change is the basic element, leading, in the end, to the transformation of
the biosphere into a noosphere in the obvious manner, above all---through the
creation and growth of the scientific understanding of our surroundings.


104. The emergence of cultural biogeochemical energy on our planet is a major
factor in its geological history.  This had been prepared for through all
geological time.  The main, decisive process here is the maximum manifestation
of the human mind.  But this is, in essence, inseparable from all
biogeochemical energy of living matter.

The life of the migration of atoms in the living process connects in a unified
whole all migrations of atoms of the biosphere's inert matter.

Organisms are alive only while the material and energetic exchange between
them and their surrounding biosphere is uninterrupted.\footnote{
	The complete absence of exchange for the latent forms of life cannot be
	considered proven, yet.  It is extremely slow---and, possibly, in a few
	cases there is no migration of atoms indeed---it could become
	noticeable only in geological time.
}  Colossal definite chemical cyclical processes of atomic migration, in which
living organisms enter as a lawful, inseparable, often main part of the
process, are being clarified in the biosphere.  These processes are constant in
geological time and, for example, the migration of magnesium atoms incorporated
in chlorophyll stretches uninterruptedly for at least two billion years through
innumerable, genetically related generations of green organisms.  Living
organisms, uninterruptedly and inseparably connected to the biosphere by such
atomic migrations, comprise a lawful part of its structure.

This must never be forgotten in the scientific study of life and in scientific
statements about any of its manifestations in Nature.  We cannot overlook the
fact that an uninterrupted connection---material and energetic of the living
organism with the biosphere, a completely definite connection, ``geologically
eternal'', which can be scientifically expressed precisely---is always present
in our every scientific approach to life and must be reflected in all of our
logical conclusions and results about it.

In moving to the study of the geochemistry of the biosphere we must, first of
all, precisely estimate the logical significance of this connection,
unavoidably entering all of our constructs related to life.  It does not
depend on our will, and cannot be excluded from our experiments and
observations, but must always be taken into account as something fundamental,
inherent in life.

The biosphere must, in this manner, be reflected in all of our scientific
statements without exception.  It must be manifest in every scientific
experiment and scientific observation---and in every thought of the human
individual, in every speculation, from which the human individual---even
thought---cannot escape.

Therefore, the human mind can be maximally expressed only with the maximum
development of the basic form of the biogeochemical energy of mankind, i.e.
with its maximum reproduction.


105. The potential for covering the surface of the whole planet by means of
reproduction of an organism of a single species is characteristic of all
organisms, since the law for reproduction is expressed in the same form for
all of them, in the form of a geometrical progression.  I have already
indicated the major significance of this phenomenon long ago,\footnote{
	See \cite{vernadsky1926biosfera-p37-38}. In the book
	\cite{vernadsky1994zhivoe-p335}; \cite{vernadsky1926etudes1}.  In the
	book \cite{vernadsky1994zhivoe-p413-424};
	\cite{vernadsky1940biogeohimicheskie-p59-83}. In the book
	\cite{vernadsky1992trudy-p75-101}.
} and I will return to it at the appropriate place in this book.

The phenomenon of covering the whole surface of the planet by a given single
species can be seen widely developed for aquatic life in the microscopic
plankton of lakes and rivers, and for a few forms of---essentially also
aquatic---microbes, from the surface layers of the planet, propagating through
the troposphere.  Among larger organisms we observe this in almost full
measure in a few plants.

This has begun to be manifest for mankind in our times.  The whole globe and
all the seas have been encompassed by him in the 20th century.  Thanks to the
success of communications, man can be in constant communication with the whole
world, cannot be solitary and get himself lost in the grandiosity of the
earth's nature anywhere.

Presently, the number of the human population on Earth has reached
unprecedented height, nearing two billion people, despite the fact that murder
in the form of war, hunger, malnourishment, constantly affecting hundreds of
millions of people, extremely diminishes the course of the process.
Negligible time from the geological point of view would be necessary, hardly
more than a few hundred years, to end these relics of barbarism.  This could
be freely done even now; the ability to end this condition is presently in the
hands of mankind, and the reasonable will will inevitably go down that path,
because it corresponds to the natural tendency of the geological process.  It
should be so all the more, since the means to do it are increasing rapidly and
almost tempestuously.  The real significance of population masses, suffering
the most from this, is irrepressibly increasing.

The number of people inhabiting the planet began increasing, say, about 15--20
thousand years ago when mankind became less influenced by food shortage in
relation to the discovery of agriculture.  Apparently it was then, say, about
10--8 thousand years ago that the first population explosion
occurred.\footnote{\cite{childe1937man-p78-79}} G.~F.\ Nikolai (in
1918--1919)\footnote{\cite{nikolai1919biologie-p54}.} attempted to estimate the
actual population increase of mankind and the development of agriculture
numerically, the actual population of the planet by mankind.  According to his
calculations, taking the total territory of the Earth, there are 11.4 people
per square kilometer, which constitutes $2.10^{-4}\%$ of the possible
population.  Considering the amount of energy received from the Sun,
agriculture allows 150 people to be sustained per $1\,{\mathrm{km}}^2$, i.e.\
for the whole Earth (land area) it must be $22.5\cdot 10^9$ units, i.e. 22--24
times more than live presently.\footnote{\cite{nikolai1919biologie-p60}.}  But
mankind acquires energy for sustenance and for living not only through
agricultural labor.  Considering this possibility, Nikolai, for example,
estimated that the Earth in the historical age started in our time, using new
energy sources, could be populated by three hexillion people ($3\cdot
10^{16}$), i.e. more than tens of millions of times more than the present
number of mankind.  These numbers must be highly increased at the present
moment, when more than 20 years have passed since Nikolai's calculations, since
mankind can, in practice, presently use sources of energy, which Nikolai could
not imagine in 1917--1919---energy, connected to the atomic nucleus.  Must now
say, more simply, that the source of energy, which is encompassed by the human
mind in the energetic age of mankind, which we are entering---is practically
unlimited.  Hence, it is clear that the cultural biogeochemical energy (§17)
shares the same characteristic.  According to Nikolai's calculations, machines
increased mankind's energy more than ten times in his time.  We cannot
presently give a more precise calculation; however, recent accounts of the
American Geological Committee\footnoteRus{американского Геологического
комитета} indicate that water power, presently in use all around the world,
reached 60 million horsepowers at the end of 1936: it increased by 160 per cent
in 16 years, mainly in North America.\footnote{\fullcite{blair1938water}.}
Thanks to that, we must already increase Nikolai's calculations more than one
and a half times.

In essence, all of these calculations about the future, expressed in a
numerical form, have no significance, since our knowledge of the energy
accessible to mankind is, we can say, rudimentary.  Of course, the energy
accessible to mankind is not an infinite amount, since it is determined by the
size of the biosphere.  The limit to the cultural biogeochemical energy is
also determined by this.

We shall see (§138) that there is also a limit to the basic biogeochemical
energy of mankind---the speed of expansion of life, the limit of mankind's
reproduction.

The speed of reproduction\footnote{
	On the speed of expansion of life see below.  See
	\foreignlanguage{french}{\cite{vernadsky1926etudes1}} in the book
	\foreignlanguage{russian}{\cite{vernadsky1994zhivoe-p413-424}\rtitle{Живое
	вещество и биосфера}, с.~413--424};
	\foreignlanguage{russian}{\cite{vernadsky1940biogeohimicheskie-p118-125}}
	in the book
	\foreignlanguage{russian}{\cite{vernadsky1994zhivoe-p437-444}};
	\foreignlanguage{russian}{\cite{vernadsky1965himicheskoe-ch20}.}
}---the magnitude $V$ considered, in essence, by Nikolai, is based on the
actually observed population of the planet by mankind in unfavorable for his
life conditions.  We shall also see, further on, that there are still unknown
to us phenomena in the biosphere, which lead to a stationary maximum quantity
of living units per hectare which can exist in a given geological age in a
given condition of the biocenosis.


106. We can record the human population on the planet with any precision only
since the beginning of the 19th century.  It is still calculated with a high
percentage of possible error.  Our knowledge has considerably increased during
the last 137 years, but can still not be considered having reached the
precision which contemporary science may require.  For earlier times the
numbers are only provisional.  Still, they are helping us in the understanding
of the occurring process.

The following data may have significance for us in that aspect.

The number of people in the Paleolithic likely reached a few million.  It is
possible that it began with one family.  However, the opposite view is also
possible.\footnote{See E.~Le Roy. [The author's note has not been found.
---Ed.]}

In the Neolithic we are likely dealing with tens of millions on the whole
surface of the Earth.  It is possible that even in historical time it did not
reach a hundred million, or that it did not exceed that number by
much.\footnote{
	\cite{weinberg1922dvuhdesyatitysyachiletiyu-p21} (assumes a
	population of 80 million at the beginning of our age).}

G.~F.\ Nikolai supposed that the human population of the planet increases by 12
million people annually for 1919, i.e. increases by, say, 30 thousand a day.
According to the critical report of the Kulischers
(1932)\footnote{\cite{kulischer1932kriegs-p135}.} the world population was 850
million in 1800 (A.~Fischer takes it to be 775 million).  We can assume its
number for the white race to be 30 million in 1000, 210 million in 1800, 645
million in 1915.  For the whole population in 1900, according to the
Kulischers---about 1,700 million, but according to A.  Hettner
(1929)\footnote{\cite{hettner1929gang-p196}}---1,564 million, and 1,856 million
in 1925, according to the same.

That number has evidently reached about two billion, more or less, at present.
The population of our country (about 160 million) comprises about 8\% of the
world population.  The world population is rapidly increasing, and, evidently,
the percentage of our population is increasing, since its growth is greater
than the average population growth.  In general, we should expect to
significantly exceed 2 billion by the end of the century.


107. The reproduction of organisms, i.e. the manifestation of biogeochemical
energy of the first type, without which there is no life, is inseparable from
man.  However, at his very differentiation from the mass of life on the
planet, man had already mastered the use of tools, even if they were very
primitive, which allowed him to increase his muscle power, and were the first
manifestation of contemporary machines, which distinguished him from the other
living organisms.  The energy by which they were powered, however, was
produced through the feeding and breathing of man's very organism.  It has
probably been hundreds of thousands of years already since man---genus
Homo,---and his predecessors mastered the use of wooden, bone, and stone
tools.  The skill of making and using those tools was being developed slowly,
in the course of many generations, skill---the mind in its first
manifestation---was being perfected.

Such tools can be observed already in the most ancient Paleolithic, 250
thousand--500 thousand years ago.

A significant part of the biosphere was living through critical times during
that period.  Apparently, a radical change---in its water and heat
regime---began already in the Pliocene, an ice age began and was developing
throughout the whole period.  We are, apparently, still living during the
dying out of its last manifestation, whether temporary or permanent is still
unknown.  We can see strong oscillations in the climate during these half a
million years; relatively warm periods---continuing for tens and hundreds of
thousands of years---replaced in the northern and southern hemispheres
periods, during which masses of ice which reached depth of up to a kilometer,
for example, in the vicinity of Moscow, moved slowly---on the historical
scale.  They disappeared a thousand and seven\footnoteTransl{
	The other English translation has seven thousand here, and notes ``Now
	we know that in the environs of Leningrad the ice has disappeared about
	12 thousand years ago.''
} years ago in the Leningrad region, and are still occupying Greenland and
Antarctica.  Apparently, Homo sapiens, or his closest predecessors, formed not
long before the onset of the ice age, or during one of its warm periods.  Man
survived the coldness during that time with hardship.  That was possible thanks
to a great discovery in the Paleolithic---the mastery of fire.

This discovery was made in one--two, possibly a few more, places and slowly
spread among the population of the Earth.  Apparently, we have a general
process of great discoveries here, where not the mass activity of mankind,
smoothing out and amending particularities, but rather the manifestation of
the separate human individuality plays a role.  We can trace that in the more
recent time and in very many cases, as we shall see later (§134).

The discovery of fire is the first case of a living organism mastering and
harnessing a force of nature.\footnote{\cite{childe1937man-p56}. Cp.:
	\cite{frazer1930myths}.}

This discovery is the foundation, as we shall now see, of all the following
increase of mankind, and of our present power.

This increase, however, took place extremely slowly, and it is hard for us to
imagine the conditions, under which it could occur.  Fire was already known to
the ancestors of the genus, or to the predecessors of that species of Hominid,
who is building the noosphere.  The latest discovery in China reveals the
cultural remains of Sinanthropus, which indicate his wide use of fire,
apparently, long before the last glaciation of Europe, a hundred thousand
years before our time.  We presently have no data of any credibility about how
that discovery was made by him.  Sinanthropus already possessed a mind, had
primitive tools, used speech, performed burial rites.  This was already a
human, but foreign to us in many morphological characteristics.  Also, the
possibility that he is one of the predecessors of the contemporary human
population of China has not been eliminated.\footnote{
	On Sinanthropus's technology, and on his use of fire see
	\cite{bogaevsky1936tehnika-p26-27}.  Pithecanthropus, who lived
	earlier, at the very beginning of the Pleistocene, hardly more than 550
	thousand years ago, also possessed fire.  Ср.:
	\cite{bogaevsky1936tehnika-p11.67}.  The use of fire by Pithecanthropus
	cannot be considered proven, yet, but is very likely.}


108. The discovery of fire is all the more remarkable because the
manifestation of fire and light emission in the biosphere had been a
relatively rare phenomenon before mankind, and had manifested mainly when
taking up a large space, in the form of cold light, in such forms as airglow,
aurora borealis, sheet lightning, stars and planets, noctilucent clouds.  The
Sun alone, the source of life, was simultaneously a bright manifestation of
light and heat, was lighting and heating the planet.

Living organisms had developed a manifestation of cold light long ago.  It
appeared in such large-scale phenomena as marine bioluminescence, usually
taking up hundreds of thousands of square kilometers, or the luminescence in
marine depths, whose significance is just beginning to be clarified.  Fire,
accompanied by high temperature, was manifested in local phenomena, rarely
taking up large spaces like volcanic eruptions.

But these colossal on the human scale phenomena, obviously, because of their
destructive force, could in no way have aided the discovery of fire.  Man had
to look for it in closer to him, and less scary and dangerous manifestations
of nature than volcanic eruptions, still exceeding mankind in their
manifestation of power.  We are only beginning to approach using them in
practice, in conditions which were inaccessible and unthinkable to Paleolithic
man.\footnote{
	Mankind has obtained superheated vapor at a ${140}^\circ C$ temperature
	as a source of power only in the 20th century with the aid of drilling
	in Larderello under Le Conte's initiative.  Still later, this method
	was greatly developed in Soffioni, in New Mexico, in Sonoma.  Parsons,
	before his death, worked on an implementable project to obtain an
	unlimited, from mankind's point of view, source of energy from the
	inner heat of the earth's crust with the aid of deep drilling.  The
	attempt to obtain energy from the cold depths of the ocean, which the
	French Academician Claude did not realize only because of criminal
	hooliganism in 1936, can be considered analogous.  Undoubtedly, we have
	in these phenomena a practically inexhaustible force in mankind's
	hands.}

He had to look for phenomena giving heat and fire in his surrounding everyday
phenomena of life; in his habitat---in the woods, steppes, among living
nature, with which he was in close (long forgotten by us) connection.  Here he
could encounter fire and heat in a safe form in numerous everyday phenomena.
These were, on the one hand, fires, the burning of living and dead matter.
They were the very sources of fire used by Paleolithic man.

He burned wood, plants, bones, that which produced fire around him without his
will.  This fire was due to two very different reasons before man's emergence.
On the one hand, lightning caused forest fires, or set dry grass on fire.
Mankind still suffers from fires caused this way.  The natural conditions in
the ice age, especially in interglacial ages, could have been even more
favorable for lightning phenomena.  There was, however, another cause which
produced fire independently of mankind.

That was the biological activity of lower organisms, which lead to fires in
dry steppes,\footnote{
	The spontaneous ignition of dry grass in the steppes, in pampas, in
	forests has sometimes been denied.  Presently the source of fires is
	almost always man, but there are cases which, it seems to me,
	undoubtedly indicate the possibility of spontaneous ignition in steppes
	under the direct action of the sun.  The cause remains unclear.  About
	such cases see \cite{popping1835reise-p398}.
	\cite{carpenter1920naturalist-p76-77}.
} to the burning of bituminous coal layers, to the burning of peat bogs, which
continued throughout a number of human generations and gave a convenient way of
obtaining fire.  We have direct indications of such bituminous coal fires in
Altai, in the Kuznetsk basin, where they occurred in the Pliocene and
post-Pliocene, but where they also occurred in historical time, and where we
still have to deal with them.  The causes of these fires are still not
completely clear, but all indications are that it is unlikely that we have
phenomena of purely chemical spontaneous combustion, i.e.  intensive oxidation
of coal fragments with oxygen from the atmosphere, or its spontaneous ignition
due to heat released during oxidation of sulphur compounds of iron in the
coal.\footnote{
	See \cite{usov1924sostav-p58, usov1933podzemnye-p34,
		obruchev1934podzemnye-p83-85}.
	J.~F.\ Hermann\footnoteRus{И.~Ф.\ Герман}, who discovered
	the Kuznetsk bituminous coal basin, already indicated these phenomena
	in 1796.  See \cite{hermann1793notice}.  Cp.
	\cite{jaworsky1933erdbrande, yavorski1932kamennougolnye}.}

The most probable source is the biochemical phenomena associated with the
biological activity of thermophilic bacteria.  We have the direct observations
of B. L. Isachenko\footnoteRus{Б. Л. Исаченко} and N. I.
Malchevskaya\footnoteRus{Н. И.  Мальчевская}\footnote{
	See \foreignlanguage{russian}{\cite{isachenko1936biogennoe}}.
} for peat bogs in recent times.

This phenomenon presently requires careful study.


109. Such regions of warm winter and summer, as well as places of outlets of
heat sources, were precious gifts of nature to Paleolithic man, who had to use
them just as they are used, or were used until recently by tribes and peoples
that we still find in a living Paleolithic stage.

Man at that time, with his great attentiveness and closeness to nature,
undoubtedly noticed such places, and must have been using them, especially in
glacial periods.

It is curious that we can observe the use of the same biochemical processes
among the instincts of animals.  This can be observed in the family of the
chickens, with the so-called incubator birds, or large-foots (Megapodiidae) of
Oceania and Australia, which make use of the heat of biological decay, i.e. of
a bacterial process, for the hatching of chicks form eggs, creating large
mounds of sand or dirt mixed with strongly rotting organic
remains.\footnote{
	See \foreignlanguage{russian}{\cite{brem1912zhizn-ptitsy}}.
}  These mounds can reach 4 meters in height, and the temperature in them
reaches no less than ${44}^\circ C$.  Apparently, these are the only birds
possessing such instincts.

It is possible that ants and termites increase the temperature of their
dwellings on purpose.

However, these are weak attempts, incomparable to that planetary revolution,
which mankind has produced.

Man has been using the products of life---dry plants---as a source of energy,
fire.  Numerous myths about its creation have been preserved and
created.\footnote{See \cite{frazer1930myths}.}  But most characteristic is the
fact that man used, for that purpose, methods which he hardly ever observed to
produce fire in the biosphere until his discovery.  The most ancient methods
were, apparently, the transformation of man's muscle power into heat (strong
friction of dry objects), and the making and catching of sparks from stones.  A
complex system for the preservation of fire was developed in the end in
everyday life a hundred, and more, thousand years ago.

The surface of the planet has been changed radically after this discovery.
Fireplaces shone, were extinguished and started everywhere, if only man lived
there.  Mankind was able, thanks to this, to survive the coldness of the
glacial period.

Man was producing fire among living nature, subjecting it to burning.  In this
way, by means of steppe and forest fires, he acquired a force which, in
comparison to that of his surrounding animal and plant world, put him above
the numerous other organisms and became a prototype of his future. Mankind has
mastered other sources of light and heat---electrical energy---only in our
time, in the 19th--20th centuries.  The planet started shining even more, and
we have found ourselves at the beginning of times, whose significance and
future still remain outside of our attention.


110. 


. . .



%\Chapter{}
\refstepcounter{chapter}

%\Chapter{}
\refstepcounter{chapter}

\Chapter{.}

151. But the contemporary state of biology and its excursions into philosophy
are also detrimental to philosophy.

The expectant attitude of the naturalist for the confirmation of philosophy
creates among philosophers the impression that precisely the
scientists\footnoteTransl{probably a typo, and should be: ``exact scientists''
[---Pav]}, proceeding from their data, accept the basic tenets of the
philosophical current of materialism about the lack of fundamental difference
between living and inert.  Vitalistic notions have remained so far in the past
in the general course of biological thought that their real significance hardly
influences large-scale work.  The overbearing majority of naturalists are far
from them.

The philosophers-naturalists, whose significance in contemporary philosophical
thought, in its global scope, is minute, receive [from the exact scientists]
what seems like firm ground, and calm their doubts.  This impacts their
creative work, which slowly dies down, and degenerates into dry, formal
scholasticism, or into verbal talmudism, especially in such cases as our
country, where dialectical materialism is the state philosophy, and is
favoured by the mighty support of government power, and by intellectual and
practical impossibility of its free criticism and of the free development of
any other philosophical views.

However, official dialectical materialism itself, being one of the many forms
of this current of philosophical thought, does not possess such freedom,
either.  And has been, meanwhile, never systematically philosophically worked
out to the end, remaining full of unclarity and unthoughtfulness.  Its official
exposition has changed more than once during the past twenty years, previous
ones were declared heretical, and new ones were created.  Our philosophers of
strict discipline, in which they work, have been obliged to obey without
objection, under the threat of persecution and material hardship, these new
ones, and to publicly repudiate their previous teachings, admitting their
mistakes.  It is easy to imagine what result follows, and how fruitfully can
one work intellectually in such a severe real environment.  As a result, a
condition very reminiscent of the condition of the orthodox church under
despotism has arisen, with the gradual downfall of lively work, work in this
area of philosophy, the exit into safe areas of knowledge, the publication of
classics, forebears; a new degeneration of thought has arisen.


152. It seems to me that for these 20 years, except the republication of old
works, which were released in the pre-revolutionary period, not a single
independent, purely philosophical work has been published, and there are not
even histories, based on primary sources, of the creation of dialectical
materialism itself.\footnoteEd{This part of the phrase is crossed out by the
author in the manuscript.}  Such decline of philosophical thought in the area
of dialectical materialism in our country, and the seemingly extensive
possibilities of its manifestation, are a consequence of the adopted
understanding of the goals of philosophy, and of the decrease of deep
philosophical work, thanks to the belief among our philosophers that a
philosophical truth, which cannot be changed and subjected to doubt any
further, has been reached.

Such an idea is, essentially, foreign to both K. Marx and F. Engels, not to
mention Feuerbach.

It was developed on Russian soil in the middle of emigration, and grew into a
state ideological influence completely unconsciously, its consequences being
unexpected for many very prominent freely thinking communists, as well.

The fight of the intellectual circles turned, in the end, imperceptibly and
unsuspectedly, into a state philosophy of the winning interpretation of
dialectical materialism.

Thanks to the strengthening of one definite current, this has been manifested
more and more clearly during the past 10 years.

As a result, we see, or we have, instead, a mass of literature of a transient
character, rooting out conscious or unconscious errors and heresies, deviations
from the officially accepted state philosophy.  On top of that, the state
philosophy itself has changed in very important nuances in the accepted
interpretation of dialectical materialism.  Such a sad state of work in our
country in the area of dialectical materialism at the presence of huge material
resources, which had never existed for any other philosophy (except for
theological ones---Catholic and Muslim philosophies in the Middle Ages), would
unavoidably come in another way, as well, thanks to many peculiarities in the
structure of state philosophy in our country.  On the one hand, thanks to the
emigration of intellectual circles, whose significance was already indicated;
and, on the other, thanks to the complexity, independent of life in our
country, of the environment, in which dialectical materialism was being
created.


153. Dialectical materialism, in the form in which it is actually manifested in
the history of thought, was never presented coherently by its authors---Marx,
Engels, and Ulyanov-Lenin.  These were prominent thinkers, and no less
prominent political activists.  Characteristic of them are a large breadth of
scientific knowledge and scientific interests, unusual for political activists.
They stood at the level of their time, but at the same time were volitional
personalities, organizers of the popular masses.  They were actively opposed
to, and regarded strongly negatively religious searches, judging them,
historically, as a force hostile, in the end, to the interests of the popular
masses and to the freedom of scientific work.  However, they, at the same time,
attributed great significance to philosophical thought, whose primacy over
scientific thought did not raise any doubt to them.

Their philosophical ideology was most closely related to their political
activity, and left an imprint on their scientific searches and understanding.
They were primarily philosophers, spokesmen for aspirations, and
organizers of the actions of the popular masses, whose social well-being---on a
real planetary basis---was the goal and meaning of their lives.  We see, by the
example of these people, a real, great impact of the personality not only on
the course human history, but, through it, on the noosphere, as well.

Part of the polemical works which their authors---Marx, Engels, Lenin,
Stalin---never intended for such a task were laid in the foundation of the
Soviet state philosophy; their statements on practical and political questions
of life, in which philosophy sometimes occupied a secondary place.  Such were,
secondly, draft notebooks, extracted from the manuscripts remaining after their
deaths, often reports and overview summaries related to the reading of
philosophers, which were never historically, scientifically, critically
published.  They were published by the scientific apparatus and with the
obeisance of believing students, and, as always under such circumstances, are
full of contradictions, and, in some cases, such as the Engels's
\rtitle{Dialectics of Nature}, the authorship of all of Engels's statements
cannot be considered proven.  A few works of Marx, and, partly, Engels, have a
different character, but they are completely insufficient for the firm
establishment of a new philosophy.  Marx' and Engels' life work was in another
domain.  Marx was a prominent scientist, who in the \rtitle{Kapital} reached
his conclusions by an exact scientific pathway, but presented them in the
language of Hegelian philosophy, independently reworked by him and Engels,
which already during their lifetimes did not (in general) correspond to current
scientific methodology and scientific searches.  The prominent mind could
permit itself such a peculiar form of presentation.

Already during Marx's lifetime---at the publication of the last volumes of his
\rtitle{Das Kapital}---such a presentation was an obvious anachronism, and it
is an even greater one in our time.  In essence, of course, what is important
is not the form of presentation of the scientific work, but rather the actual
methodology, by which what is presented has been reached.  The form of Marx's
presentation misleads the reader into thinking that what is presented was
reached by a philosophical pathway.  It is, in reality, only presented that
way, but was, in fact, reached by the exact scientific method of the historian
and economist-thinker, who Marx was in his scientific work.

It turned into a complete anachronism, since it was transferred from the area
of political economy and history into the area of natural and exact sciences.
This transfer, which can be observed in the works of both Marx and Engels,
acquired an extremely peculiar character with their epigons, having become the
state philosophy of a large and strong nation, most closely related to the
International.

Thirdly, the situation was worsened by the fact that the authors of these
philosophical searches were people, either actually exercising dictatorial
power in an unprecedented depth and degree, and considering the philosophical
ideology of dialectical materialism as the basis of their political and
practical activity, or people, such as Marx and Engels, who are not subject to
free criticism in our country for the same reason.  Their conclusions are, in
fact, accepted as impeccable dogma, defended by the full mechanism of
government power.

The stagnation of philosophical thought here, and its transformation into
fruitless scholasticism and talmudism, opulently blooming against that
background, is a direct consequence of this state of affairs.

This, in essence, great historical phenomenon was prepared in our country by
deeply-rooted submissiveness---unchanged during all the transformations of the
form of government---to the state religion.  The official Orthodoxy in the
Tsardom of Russia, as well as in the Russian Empire, prepared the ground for
the official philosophy, which replaced it, and which has acquired the clear
form of official religion with all of the consequences from that.

154. This, however, is, historically and in essence, only the everyday side of
the matter.  The ideology and its associated belief at its foundation are far
more important.

Dialectical materialism, in sharp contrast to contemporary forms of philosophy,
is extremely distant from philosophical scepticism.  It is convinced that a
universal method rules---an infallible criterion of philosophical and
scientific truth.  This is the effect of the temperament of its founders Marx
and Engels, who succeeded, thanks to their joining the still alive at that time
Hegelian philosophy, to impart to their scientific achievements the vibrantly
active form of faith, and not only of a philosophical doctrine---to create a
political force, able to move the masses and vividly manifest itself in the
\rtitle{Communist Manifesto} of '48---in a brilliant and profound work,
reflecting the age of the middle of the last century, when the primacy of
philosophy over science dominated ideologically Euro-American civilization.

In contrast to other forms of materialism, with which it is in fundamental
disagreement, dialectical materialism is closely related in its genesis and in
the basis of its formulations with idealism in its Hegelian form.

It is far from clear, whether it is possible to regard it as free from the
influence of such history, and to attribute it completely to the philosophical
current of materialism.

As far as I know, this question is historiographically unresolved, and in the
manifestation which materialism has in our country, its idealistic basis is
strongly emphasized, whereas its materialistic one is an outer appearance.

But this is a debatable area, far from my interests, and from my knowledge, and
I would not concern myself with it, if the sharp distinction between the
philosophical current of materialism and dialectical materialism did not become
completely clear in our country in the aspect which most concerns the
naturalist and seriously affects scientific work in our country.

Materialistic philosophy was evidently distinct---and that is where its force
lied---from the other philosophical currents of modern times, in the fact that
it did not conflict with science, was completely based on its achievements, as
far as possible.  It generalized and developed them.  In essence, it continued
that great movement, which developed in the 17th--18th centuries on the basis
of the new science, the new philosophy, and the new ways of everyday life and
technologies, which were created at that time.

Materialism, in essence, was striving to become a scientific philosophy, or a
philosophy of science.  It did not succeed in practice, since in its logical
conclusions, being part of the philosophy of the Enlightenment from the end of
the 18th century, when it clearly occupied a place on the historical stage for
the first time, it quickly fell behind the science of the times.

But in the aspect concerned in this book, what is important is not the success,
or failure of materialism in its historical manifestation during the age of its
flourishing at the end of the 18th century, and in the 1860s, but the
foundation of its ideology, which always recognized the primacy of science
above philosophy.  It considered everything proven by science as obligatory for
itself.

The dialectical materialism, created by Marx and Engels, did not accept that,
and, in that, sharply distinguished itself from all forms of philosophical
materialism, and, from that standpoint, did not differ at all from idealistic
Hegelianism.

For that very reason, it is also clearly distinct from philosophical
scepticism, which accepts the realistic worldview, as it is manifested
scientifically, as the only possibility, and does not recognize, in comparison,
either religious, or philosophical views on an equal basis.  Philosophical
scepticism, in contrast to philosophical materialism, does not recognize the
scientific view of reality as its complete view, taking into account the
increase of scientific knowledge, and the imperfections of human reason.  But
for it the scientific achievements at a given historical moment, and at a given
form of the human brain have the character of the most precise achievement of
reality.  Dialectical materialism does not proceed from scientific data, is not
limited to their boundaries, is not based on them, but is striving to change
and develop them, adapting them to its views, which have as a basis the laws of
Hegelian dialectics.  It seems to me that this dialectics is so closely related
to the whole philosophy of Hegel that through it foreign, from the standpoint
of materialism, formulations enter into the spiritual environment of
materialism---mystical, distorting to it, such as, for example, the
manifestation of dialectics in nature, or in the present case, speaking
scientifically, in the biosphere.

The introduction of the dialectics of nature in the philosophical purview of
our country, in its official philosophy, during our time of great increase, and
significance of science---is a remarkable historical phenomenon.

This has been the form of the post-mortem influence of the works of Marx and
Engels, based on faith---officially---but not expressed philosophically, or
scientifically, etc.\ [by them].


155. Effectiveness, i.e. the equal significance of methodological thought and
the instructions of the philosophers-dialecticians for current scientific work,
is strongly underscored in our philosophical literature, and is introduced into
science through the agency of government power.

The philosophers-dialecticians are convinced that they can aid current
scientific work with their dialectical method.

They believe in its significance for science, but the manifestation of that
belief in reality contradicts it.

It appears to me that this is a misunderstanding.  No philosophy has played, or
plays, such a role in the history of thought.  No philosopher can instruct the
scientist in the pathway to take in the methodology of scientific work,
especially in our times.  The philosopher is not capable of precisely
encompassing the complex problems, whose solutions stand today before the
naturalist in one's current work.  The methods of scientific work in the area
of experimental sciences and descriptive natural sciences, and the methods of
philosophical work, even in the area of dialectical thought, are expressly
different.  It seems to me, the two lie in different domains of thought, as far
as we are dealing with concrete natural phenomena, i.e. with empirically
established facts, and empirical generalizations built upon scientific facts.
It seems to me that the issue here is so clear that no argument is necessary.
Our philosophers-dialecticians must not interfere with this area of scientific
knowledge for their own benefit.  Here, also, their attempt is doomed to
failure from early on.  Here they are fighting with science on its native
terrain.

Science lived through a similar interference of religious thought and religious
constructs, erroneous at their roots, during the age of the Renaissance, during
the 17th--19th centuries.  Though the fight here is not yet over, hardly
anybody would deny that victory has remained on the side of science, that the
majority of religious constructs of that type remained in the past, or are
being reconstructed in their essence, reinterpreted, are shifting from the area
of reality into that of personal belief and interpretation.  The historical
experience was not taken into account by the official philosophers of our
country, and they, in their squareness and insufficient scientific literacy,
entered into a sharp conflict with scientific thought and work, which are
correctly placed ideologically high in our country---on an equal level with
dialectical materialism---at the foundation of our system of government.

The weakness of placing ``dialectical materialism'' at such a height,
unavoidably impacts its real power in nation building, does not correspond to
reality, and unavoidably proves to be transient.

Conflicts with the actual needs of life are beginning, which must unavoidably
have those consequences, which came into being \dots\ supreme
\dots\footnoteEd{Illegible in the manuscript.} in the old Christian nations.


156. I have collided with this kind of circumstances in my scientific work many
times, and have even mentioned the struggle of my predecessors in scientific
knowledge from the past century in public statements.

In 1934 little-educated philosophers, heading the planning of scientific work
of the former Geological Committee\footnoteRus{Геологический комитет},
erroneously attempted to prove, by means of dialectical materialism, that the
determination of geological age by means of radioactivity is based on erroneous
theses---dialectically unproven.  They thought that the facts and empirical
generalizations that radiologists relied upon were dialectically impossible.
They were joined by a few geologists, occupying themselves with philosophy, and
heading the scientific leadership of the Committee.  They held up my work by
one-two years, because the Radium Institute\footnoteRus{Радиевый институт},
which I headed, was completely unable to get in touch with the work of the
Committee geologists, and to put the investigations on a solid basis.  In the
end, after an uncareful statement at the public session of the Committee by the
Vice Scientific Director\footnoteRus{заместителя директора по научной части}
professor M.  M.  Tetyaev\footnoteRus{М. М. Тетяев}, a prominent geologist,
publicly indicating the incompatibility between dialectical materialism and the
conclusions of radiologists, it was possible to achieve a, now public,
discussion on this subject.  It was possible to do so, because the whole
radiological work of the Committee was under attack by his statement.  I was
able to intervene in my role as an Acting Chairman\footnoteRus{и.~о.\
председател} of the Committee on Geological Time\footnoteRus{Комитета по
геологическому времени}, having been elected at the Soviet Union Radiological
Conference\footnoteRus{Всесоюзной Радиологической конференцией}, and to acquire
a public debate of this question.  This took place under my chairmanship at the
premises of the Geological Commitee, where I placed the condition that we, as
insufficiently competent in dialectical philosophy, would only address the
scientific side of the phenomenon.  The striking ignorance of the basic facts
and achievements in the area of radiogeology of all philosophers and many
geologists became undeniably clear to all at that session, attended by a few
hundred geologists and philosophers.  We were able to freely develop our work
to a large degree thanks to the fact that the philosophical leaders of the
Geological Committee soon proved to be heretics according to the official
interpretation of dialectical materialism, and were excluded from the
Committee.  However, they still did harm---weakened our scientific work by a
few years.

The phenomenon which was manifested here---errors in the interpretation of
dialectical materialism by official representatives of the philosophy---is an
everyday and widespread phenomenon of our life.  There are a few philosophers,
whom it didn't suit to retract the philosophical theses set forth by them,
which has been explained by an unconscious mistake, or a conscious one, by a
hidden departure from the official philosophy, or, even, by a conscious
interference with the government.  The wide manifestation of this phenomenon,
totalling hundreds of our philosophers-dialecticians, indicates the clear to
every scientist difficulty in the application of the dialectical method in the
current scientific environment.  For, as is clear from §153, there has been not
one prominent thinker from among the founders of dialectical materialism
throughout the historical course of its development, who has given a complete
treatment of this philosophy, thought through to the end.  It has been created
by them in the dust of fights and polemics, from case to case.

None of them has made a complete presentation, and the attempts by less
prominent thinkers, unavoidably proved to be ephemeral.  Errors were found in
them, they were revoked from circulation, one was to never refer to them.  That
continued tens of times, and there remained no presentation, which could be
considered firm.  The present official presentation of both dialectical
materialism, and of the history of the Communist Party, whose ideology this is,
is dated 1936--1937, and there is no certainty than in a year or two they would
not require new reworking.

I have had the occasion to, also, encounter other manifestations of this
scientific environment.  Inexplicably, the Kant-Laplace hypothesis and the
acceptance of the possibility of abiogenesis were connected to dialectical
materialism, and their negation was considered unacceptable from a dialectical
standpoint.  Such a presentation met censorial difficulties.  Already in 1936
in my report \rtitle{On the Problems of Biogeochemistry}, I ran into objections
of that kind at the session of the Academy.  And I was able to establish the
presently unscientific character of the Kant-Laplace hypothesis, and its
incompatibility with radiogeological data the next year in my official speech
at the International Geological Congress\footnoteRus{Международном
геологическом конгрессе} to the tacit agreement of our geologists, including
those considering themselves dialecticians.

In this case the question is not about the interference of dialectical
materialism with the scientific work of the naturalist in the manner indicated
earlier.

Principally, the naturalist cannot deny the correctness and usefulness of the
interference of philosophers in one's scientific work in many cases, when what
is being dealt with are scientific theories, hypotheses, generalizations of a
non-empirical character, cosmogonic constructs.  Here the naturalist
unavoidably treads upon philosophical terrain.

Even here scientific thought finds itself in a condition, which interferes with
its correct scientific work, in our country.  In this case, our scientific
thought conflicts with an obligatory philosophical dogma, with a definite
philosophy, which, as we have seen, has no firm presentation.  This dogma, with
the lack of free scientific and philosophical investigation in our country,
with the extreme centralization of advance censorship, and all means of
dissemination of scientific knowledge---by way of printed or spoken word---in
the hands of government power, is accepted as obligatory for all, and is
introduced in popular life through the full power of government.

\begin{flushright}
								   1936--1938.
\end{flushright}



\backmatter

\printbibliography

\end{document}
